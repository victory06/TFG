% !TeX root = ../libro.tex
% !TeX encoding = utf8

\chapter{Implementación}
El código se ha desarrollado usando el software \textit{Jupyter Notebook} en el lenguaje \textit{Python}. Se estructura en seis partes:
\begin{itemize}
  \item Exploración con datos de tríos: En esta primera exploración se han utilizado sólo los datos de la familia (sin los datos de control) para la experimentación.
  \item Primeros datos de control: Se han utilizado los padres del conjunto de tríos y se han juntado con los primeros datos de control para repetir la experimentación.
  \item Segundos datos de control: Padres y segundos datos de control para verificar los resultados obtenidos con los primeros controles.
  \item Diferencias entre ambos controles: Ambos conjuntos de datos de control para detectar sus diferencias.
  \item Selección de variables: Apartado con selección de variables, reducción de dimensionalidad y reentrenamiento de modelos.
  \item Diferencias de sesgo: Apartado con dos tests para comprobar las diferencias entre los conjuntos de datos debidas al sesgo del laboratorio.
\end{itemize}
\section{Software desarrollado}
En esta sección se repasarán los paquetes que han sido necesarios incluir para la realización del código en \textit{Python}.\\
Los paquetes que se han utilizado en el trabajo son los siguientes:
\begin{itemize}
\itemsep 0em 
  \item \textit{scikit-learn} \cite{scikit-learn}: El paquete más utilizado. Se han usado distintos clasificadores, métricas y preprocesado de \textit{datasets}.
  \item \textit{pandas} \cite{mckinney-proc-scipy-2010}: Librería utilizada para la representación de datos.
  \item \textit{matplotlib} \cite{Hunter:2007}: Para la creación de diferentes gráficos
  \item \textit{scipy} \cite{scipy}: Para los tests de Wilcoxon y el T-Test.
\end{itemize}

\section{Exploración de los datos}
En las cuatro primeras secciones del código, se ha realizado una exploración de los datos sin ningún tipo de selección de variables o reducción de dimensionalidad.
En la representación de los datos se ha usado un \textit{dataframe} del paquete \textit{pandas} \cite{mckinney-proc-scipy-2010}, dejando como nombre de las filas el ID del individuo correspondiente y como nombre de las columnas o variables, se ha establecido el nombre del SNP, tanto para el alelo A como para el B.
\subsection{Conjunto de datos de tríos y primer control}
Antes que nada se ha realizado un estudio de la base de datos de tríos en el que no esperamos encontrar diferencias significativas pero nos dará una idea de cómo son los datos y cómo llegan a comportase los modelos.

Para empezar, se han leído para sólo los datos de tríos mediante el paquete \textit{pandas} \cite{mckinney-proc-scipy-2010} para leer tanto los datos como los nombres de las columnas. Los nombres de las columnas (nombres de las variables) se han repetido para que las parejas de columnas de valores de SNPs tengan el mismo nombre y se ha añadido tanto el Género como el ID familiar para seguir el formato de los datos explicado en el \autoref{ch:base-datos}. Para los datos, se han eliminado todas las columnas de IDs excepto el familiar debido a que si no se eliminaran, los modelos harían una predicción perfecta sólo basándose en si el sujeto tiene ID de padre y madre (es decir, es un hijo y es afectado) o no (es padre y es sano). Se ha dejado el género para que el modelo lo tenga en cuenta pues en la herencia genética el género es influyente, así como el ID familiar para que se relacionen los tres individuos de una familia.\\
Después se ha separado en conjuntos train (entrenamiento) y test usando un 25\% de los individuos para el train y un 75\% para el test y se han contabilizado cuántos individuos hay en cada conjunto quedando 438 sanos y 210 afectados en el test y 1290 sanos y 654 afectados en el conjunto de entrenamiento, con un total de 1944 instancias en train y 648 en test.\\
También para realizar un acercamiento más visual a los datos se han representado las intensidades de dos SNPs. En el eje de abscisas está el valor de la primera columna A para el SNP y en el de ordenadas el valor de la segunda columna B. No se ha modificado el valor de las intensidades de ninguna forma, están directamente representadas en la gráfica. Podemos ver dos colas, dos líenas en la gráfica que representan los individuos homocigóticos y heterocigóticos. También podemos apreciar que las intensidades no están en forma de punto sino de línea reflejando la naturaleza exponencial de la \textit{Polymerase Chain Reaction} (PCR), mostrando inviduos con intensidad total ($intensidad_{AB}=intensidad_{A}+intensidad_{B}$) muy baja y otros mucho más alta. La gráfica se ha dibujado mediante la librería \textit{matplotlib} \cite{Hunter:2007}.

\begin{figure}[H]
\centering
\begin{minipage}{.5\textwidth}
  \centering
  \includegraphics[width=.8\linewidth]{trios_snp1}
  \captionof{figure}{SNP mitog6735a en tríos.}
  \label{fig:snp1-trios}
\end{minipage}%
\begin{minipage}{.5\textwidth}
  \centering
  \includegraphics[width=.8\linewidth]{trios_snp2}
  \captionof{figure}{SNP mitog16392a en tríos.}
  \label{fig:snp2-trios}
\end{minipage}
\end{figure}

Visualmente se podría decir que es difícil ver diferencias entre padres e hijos, lo cual es lógico pues entre padres e hijos la diferencia genética es muy baja.\\\\

El siguiente paso fue intentar predecir si los padres iban a tener un hijo afectado quitando los individuos hijos de la base de datos de tríos, marcando los padres como afectados y añadiendo los individuos de control al conjunto marcados como sanos. También es necesario borrar dos SNPs, cuatro variables en el conjunto, que hay de más en los datos de tríos. En concreto son los SNPs mitoc10874t y mitog10590a.\\
El número de instancias total que queda en el conjunto es de 4427 y 273 variables. Se ha partido en conjuntos de entrenamiento (75\%) y test (25\%) quedando 3320 instancias en el entrenamiento con 2040 individuos sanos y 1280 padres de individuos afectados y en el test 1107 con 659 individuos sanos y 448 padres de afectados.\\
Visualmente se han representado dos SNPs con los valores del alelo A en el eje x y los del alelo B en el y para cada individuo.

\begin{figure}[H]
\centering
\begin{minipage}{.5\textwidth}
  \centering
  \includegraphics[width=.8\linewidth]{control1_snp1}
  \captionof{figure}{SNP mitoa3721g primer control.}
  \label{fig:snp1-control1}
\end{minipage}%
\begin{minipage}{.5\textwidth}
  \centering
  \includegraphics[width=.8\linewidth]{control1_snp2}
  \captionof{figure}{SNP mitoc6046t primer control.}
  \label{fig:snp2-control1}
\end{minipage}
\end{figure}

Se puede ver a simple vista que las diferencias van a ser más fáciles de encontrar que en la base de datos de tríos.

\subsection{Segundo conjunto de control}
En el caso de los segundos datos de control se ha juntado con los datos de los tríos de la misma forma que con el primer control: se han eliminado los individuos afectados (hijos) de los datos de familias y se han marcado a los padres como afectados. Después se han eliminado los SNPs redundantes de los padres y se han juntado con estos segundos datos de control. Por último, se ha separado en conjuntos de entrenamiento y test en un 75\% y 25\% respectivamente quedando de las 4648 instancias totales, 3486 para el conjunto \textit{train} (2211 sanos y 1275 padres de afectados) y 1162 para el test (con 709 sanos y 453 padres de afectados). Hay un desbalanceo de datos con 2920 individuos sanos y 1728 padres de afectados.\\
Se han representado dos SNPs con sus dos alelos A y B en el eje de abscisas y ordenadas respectivamente:

\begin{figure}[H]
\centering
\begin{minipage}{.5\textwidth}
  \centering
  \includegraphics[width=.8\linewidth]{control2_snp1}
  \captionof{figure}{SNP mitot3198c segundo control.}
  \label{fig:snp1-control1}
\end{minipage}%
\begin{minipage}{.5\textwidth}
  \centering
  \includegraphics[width=.8\linewidth]{control2_snp2}
  \captionof{figure}{SNP mitot6222c segundo control.}
  \label{fig:snp2-control1}
\end{minipage}
\end{figure}

Los datos visualmente presentan cierta separación.

\subsection{Ambos datos de controles}
Para comprobar las diferencias de sesgo que haya podido haber por la estracción de los datos en laboratorio, se ha hecho también un estudio sobre ambos datos de control. Se ha leído de la misma forma que los anteriores apartados pero esta vez no hay SNPs redundantes y se han separado en conjuntos de entrenamiento y test en un 75\% y 25\%. También se ha marcado con la etiqueta 0 los datos del primer control y con 1 los del segundo.\\
Esta vez los datos están más balanceados quedando de los 5619 individuos totales, 2920 son del segundo control y 2699 del primero. Para el test quedan 714 del primer control y 691 del segundo y para el entrenamiento, 1985 del primer control y 2229 del segundo control. Se han representado de nuevo dos de los SNPs:

\begin{figure}[H]
\centering
\begin{minipage}{.5\textwidth}
  \centering
  \includegraphics[width=.8\linewidth]{controles_snp1}
  \captionof{figure}{SNP mitog1440a dos controles.}
  \label{fig:snp1-control1}
\end{minipage}%
\begin{minipage}{.5\textwidth}
  \centering
  \includegraphics[width=.8\linewidth]{controles_snp2}
  \captionof{figure}{SNP mitoc5264t dos controles.}
  \label{fig:snp2-control1}
\end{minipage}
\end{figure}

Los controles, como se puede comprobar en un primer vistazo, están bien diferenciados, lo que apunta a diferencias de sesgo debidas a la extracción de los datos en laboratorio.

\subsection{Parámetros de los modelos empleados}
Describimos en esta sección los parámetros utilizados para cada modelo.
\subsubsection{Modelo K-NN}
Se ha utilizado la implementación del paquete \textit{scikit-learn} \cite{scikit2021knn} con los parámetros por defecto:
\begin{itemize}
\itemsep 0em 
  \item Número de vecinos: 5, es un modelo 5-NN.
  \item Pesos: Uniforme, todos los vecinos tienen la misma importancia.
  \item Métrica: Se usa la de Minkowski.
  \item p: 2, se usa la distancia euclídea.
\end{itemize}


\subsubsection{Regresión Logística}
Para la regresión logística (RL) he establecido que los pesos estén balanceados para contrarrestar el gran desbalanceo de los datos (hay el doble de padres, no afectados, que de hijos, afectados). Los parámetros utilizados han sido los siguientes \cite{scikit2021lr}:
\begin{itemize}
\itemsep 0em 
  \item Regularización: L2, regularización de Ridge.
  \item C: El valor inverso de la fuerza de regularización, establecida a 1.
  \item Optimización: "lbfgs", limited-memory BFGS.
  \item Maximo de iteraciones para que el optimizador converja: 100.
  \item Tolerancia: $10^{-4}$, la tolerancia para el criterio de parada.
\end{itemize}

\subsubsection{Árboles de decisión}
Estableciendo de nuevo los datos balanceados para reducir el desbalanceo que presentan estos conjuntos de datos, los parámetros establecidos para el modelo han sido los siguientes \cite{scikit2021dt}:
\begin{itemize}
\itemsep 0em 
  \item Criterio para medir la calidad de una división: Gini.
  \item Elegir la mejor división (parámetro \textit{splitter}=\textit{best})
  \item Mínimo 2 muestras para dividir un nodo
  \item Mínimo una muestra para ser un nodo hoja
\end{itemize}

\subsubsection{Máquinas de soporte vectorial}
Se han entrenado dos versiones de este modelo, para el conjunto de datos de tríos se han establecido los pesos balanceados y se han usado los valores por defecto y para los primeros datos de control se han cambiado vari


\endinput
%------------------------------------------------------------------------------------
% FIN DEL CAPÍTULO. 
%------------------------------------------------------------------------------------









